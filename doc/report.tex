%%%%%%%%%%%%%%%%%%%%%%%%%%%%%%%%%%%%%%%%%
% Simple Sectioned Essay Template
% LaTeX Template
%
% This template has been downloaded from:
% http://www.latextemplates.com
%
% Note:
% The \lipsum[#] commands throughout this template generate dummy text
% to fill the template out. These commands should all be removed when 
% writing essay content.
%
%%%%%%%%%%%%%%%%%%%%%%%%%%%%%%%%%%%%%%%%%

%----------------------------------------------------------------------------------------
%	PACKAGES AND OTHER DOCUMENT CONFIGURATIONS
%----------------------------------------------------------------------------------------

\documentclass[12pt]{article} % Default font size is 12pt, it can be changed here

\usepackage{geometry} % Required to change the page size to A4
\geometry{a4paper} % Set the page size to be A4 as opposed to the default US Letter

\usepackage{graphicx} % Required for including pictures

\usepackage{float} % Allows putting an [H] in \begin{figure} to specify the exact location of the figure
\usepackage{wrapfig} % Allows in-line images such as the example fish picture

\usepackage{lipsum} % Used for inserting dummy 'Lorem ipsum' text into the template

\usepackage{tikz}
\usetikzlibrary{arrows,positioning,shapes}

\linespread{1.2} % Line spacing

%\setlength\parindent{0pt} % Uncomment to remove all indentation from paragraphs

\graphicspath{{Pictures/}} % Specifies the directory where pictures are stored

\begin{document}

%----------------------------------------------------------------------------------------
%	TITLE PAGE
%----------------------------------------------------------------------------------------

\begin{titlepage}

\newcommand{\HRule}{\rule{\linewidth}{0.5mm}} % Defines a new command for the horizontal lines, change thickness here

\center % Center everything on the page

\textsc{\LARGE Project Report}\\[1.5cm] % Module name
\textsc{\Large EE5907}\\[0.5cm] % Major heading such as course name
\textsc{\large Pattern Recognition}\\[0.5cm] % Minor heading such as course title

\HRule \\[0.4cm]
{ \LARGE \bfseries Handwrite OCR with Bayes and KNN Method}\\[0.4cm] % Title of your document
\HRule \\[1.5cm]

\begin{minipage}{0.4\textwidth}
\begin{flushleft} \large
\emph{Student:}\\
SHI \textsc{XUDONG} % Your name
\end{flushleft}
\end{minipage}
~
\begin{minipage}{0.4\textwidth}
\begin{flushright} \large
\emph{Matric No:} \\
A0109682Y \textsc{} % matriculation number
\end{flushright}
\end{minipage}\\[0.5cm]
~
\begin{minipage}{0.4\textwidth}
\begin{flushleft} \large
\emph{Email:}\\
shixudongleo@gmail.com \textsc{} % email
\end{flushleft}
\end{minipage}
~
\begin{minipage}{0.4\textwidth}
\begin{flushright} \large
\emph{Tel:}\\
(+65) 85014413 \textsc{} % email
\end{flushright}
\end{minipage}\\[4cm]





{\large \today}\\[3cm] % Date, change the \today to a set date if you want to be precise

%\includegraphics{Logo}\\[1cm] % Include a department/university logo - this will require the graphicx package

\vfill % Fill the rest of the page with whitespace

\end{titlepage}

%----------------------------------------------------------------------------------------
%   Abstract
%----------------------------------------------------------------------------------------
\section{Abstract}
This report presents the result of handwritten digit recognition on well-known image database using feature extraction and classification techniques. The test database is MNIST. The features include raw pixel intensity value, gradient feature, profile structure feature \cite{liu2002handwritten}, peripheral feature and moment-based features. The comparision of feature vectors shows that gradient feature from gray-scale image mostly yields the best performance\cite{liu2004handwritten}. In addition, classification methods are compared among parametric and nonparametric classifiers.  


%----------------------------------------------------------------------------------------
%	INTRODUCTION
%----------------------------------------------------------------------------------------

\section{Introduction} % Major section



%----------------------------------------------------------------------------------------
%	Methodology 
%----------------------------------------------------------------------------------------

\section{Methodology} % Major section



%----------------------------------------------------------------------------------------
%	Experimental Result 
%----------------------------------------------------------------------------------------

\section{Experiments} % Major section

%----------------------------------------------------------------------------------------
%	Conclusion
%----------------------------------------------------------------------------------------

\section{Conclusion} % Major section



%----------------------------------------------------------------------------------------
%	MAJOR SECTION X - TEMPLATE - UNCOMMENT AND FILL IN
%----------------------------------------------------------------------------------------

%\section{Content Section}

%\subsection{Subsection 1} % Sub-section

% Content

%------------------------------------------------

%\subsection{Subsection 2} % Sub-section

% Content



%----------------------------------------------------------------------------------------
%	BIBLIOGRAPHY
%----------------------------------------------------------------------------------------
\bibliographystyle{plain}
\bibliography{handwirte_ocr}

%----------------------------------------------------------------------------------------

\end{document}